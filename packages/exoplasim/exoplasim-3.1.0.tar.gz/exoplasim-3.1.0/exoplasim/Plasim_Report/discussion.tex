In the previous chapters the performance of the Planet Simulator is shown by means of the energetics, the seasonal and annual mean climates of state variables, eddy fluxes and global surface climates. The most important results are now briefly summarised:\\
(i) The global energetics of PlaSim, expressed in the energy and water budgets and the Lorenz energy cycle, show good agreement with observations (\citet{Kiehl1997}, \citet{Trenberth2006} and \citet{Peixoto1993}) and only smaller deviations.\\  
(ii) The seasonal mean climate of the mean sea level pressure shows only small deviations compared to ERA concerning the position and strength of the main pressure systems. Likewise, the general wind pattern shows small deficiencies in the position and the strength of the jets and, although they are both underestimated, PlaSim shows a good structural agreement in terms of the development of the main features of the mass streamfunction and the velocity potential. Larger deviations occur in the temperature and radiation fields, as too less surface and top solar radiation in the winter polar regions are leading to a systematic cold bias there. This is partly due to a deficient prescribed sea ice-distribution, which is extending too far south. This accounts already for half of the total bias, so that the too strong minimum in Fig. \ref{img:surftemp} a in about 80° N is strongly reduced. This new sea-ice distribution is consequently implemented in the new version of the Planet Simulator.\\
Presentation of the global precipitation field shows an even more realistic pattern than in ERA, as the precipitation is strongly overestimated in ERA, mainly in the tropics \citep{Hagemann2005}. Consequently, the P-E-field shows larger differences between PlaSim and ERA as well due to stronger precipitation maxima in ERA.\\ 
(iii) The winter mean of the geopotential height reveals a too strong zonality leading to a deficient stationary wave structure  in PlaSim. Both bandpass filtered maxima over the Pacific and the N-Atlantic are overestimated and the latter is less extended and slightly shifted to the east and south, compared to ERA and \citet[p. 83]{Roeckner1996}. Moreover, the peaks of the wavenumber-frequency spectra of the 500 hPa geopotential height are underestimated by PlaSim as well.\\
(iv) The distribution of the zonally averaged fluxes in PlaSim reveals an opposite picture in PlaSim with a strongly overestimated heat flux and an underestimated momentum flux. This leads to an overall underestimation of the EP-flux divergence and to a deficient tilt of the EP-flux arrows. The stationary wave activity flux shows good structural results but the major wavetrains are strongly underestimated and slightly shifted to the west (east) over the N-Pacific (N-Atlantic). The cells of the residual meridional circulation are less pronounced in PlaSim, and their strength is strongly underestimated, compared to ERA and \citet[p. 325]{Holton1992}.\\
(v) PlaSim shows the basic cyclone climatology patterns of both hemispheres, but the density peaks are shifted southward (northward) in the N-Atlantic (N-Pacific) in both seasons over the NH. During JJA, both peaks are stronger and too zonally oriented, compared to ERA. Moreover, the maximum over the Denmark Strait is overestimated and those over Arabia are strongly underestimated by PlaSim. Over the SH, there is a zonal band of cyclonic activity around the Antarctic, which is too extended in PlaSim in summer and winter. The SH-winter maxima are located in the Atlantic and Indian Ocean near 60° S in both PlaSim and ERA. The Pacific-maximum, which should be located further south, is shifted about five to ten degrees to the north in PlaSim. During NH summer, weaker density maxima occur, as well as a splitted pattern in the Tasman Sea with lower densities further north. PlaSim does not show this split but similar values compared to ERA. In the tropics, PlaSim correctly shows the region of large TC activity, but slightly underestimates the total number of tropical cyclones.\\   
(vi) Finally, PlaSim clearly captures the five main climates of the Koeppen climatology and the Budyko-Lettau dryness ratio.
Compared to ERA, larger differences occur over the northern parts of N-America and Asia and smaller parts of S-America with PlaSim simulating a slightly lower dryness ratio, i.e the region of humid savanna to forest is more extended than in ERA. The desert regions over N-Africa and Australia are less extended as PlaSim equally underestimates the dryness ratio there. Although the main characteristics of the annual mean Bowen ratio are shown by PlaSim, compared to ERA, stronger differences occur over almost all continental regions with PlaSim partly strongly underestimating the Bowen ratio.


























