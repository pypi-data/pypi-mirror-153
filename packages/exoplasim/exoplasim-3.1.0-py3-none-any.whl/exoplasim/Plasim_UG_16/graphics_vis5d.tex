\begin{quote}
{\it ``Vis5D is a system for interactive visualization of large 5-D
gridded data sets such as those produced by numerical weather
models. One can make isosurfaces, contour line slices, colored slices,
volume renderings, etc of data in a 3-D grid, then rotate and animate
the images in real time. There's also a feature for wind trajectory
tracing, a way to make text annotations for publications, support for
interactive data analysis, etc.''}
\par
\end{quote}
\begin{flushright}
from the Vis5D home page,\\
{\url{http://www.ssec.wisc.edu/~billh/vis5d.html}}
\end{flushright}
\par
\noindent This powerful visualisation tool together with its documentation is
available through the above home page. Vis5D uses its own data format
which makes it necessary to transform your data. Depending on their
format
% (see also section \ref{Filestructure}
 and the flowchart on
{\url{http://puma.dkrz.de/puma/download/map/}} you have the following
choices: If

\begin{itemize}
\item{{\it your data is raw \verb#PUMA# output,}\\
	you need to process it with the \verb#pumaburner#
	postprocessor (see section \ref{Pumaburner}) in order to
	transform it to either
	\verb#NETCDF#} (option \verb#-n# or namelist parameter
	\verb#NETCDF=1#) or \verb#GRIB# (option \verb#-g# or namelist
	parameter \verb#GRIB=1#) and proceed from there.
\item{{\it your data is in \verb#SERVICE# format,}\\
	you need to convert it to either \verb#GRIB#, for
	instance with the \verb#PINGO#s:
\begin{quote}
 \verb#grb copy2 data.srv data_with_grib_metainfo.grb output.grb#,
\end{quote}	
	or \verb#NETCDF#,
	using the program \verb#puma2cdf#, which is available with the
	\verb#PUMA# postprocessing tools. Despite of its name this
	program cannot process raw \verb#PUMA# output but takes
	\verb#SERVICE# format as input. It can as well be called as
	\verb#srv2cdf# which changes its behaviour: oddities of model
	output such as the existence of February, $30^{th}$ are
	then no longer removed.	Once the format is changed proceed from there.}
\item{{\it your data is in \verb#NETCDF# format,}\\
	it can easily transformed to \verb#Vis5D#'s native format by
	means of the program \verb#cdf2v2d#, which is available with
 	the \verb#PUMA# postprocessing tools.}
\item{{\it your data is in \verb#GRIB# format,\\}
	you can find a transformation tool named \verb#Grib2V5d# at\\
	\verb#<http://grib2v5d.sourceforge.net># which offers various
	practical features.}
\end{itemize}
\noindent Once the conversion to \verb#Vis5D#'s native format is
	achieved please follow the instructions from the \verb#Vis5D#
	documentation or, if \verb#Vis5D# is already installed on your
	system, try finding your own way by typing:
\begin{quote}
\verb#vis5d my_data.v5d#
\end{quote}

