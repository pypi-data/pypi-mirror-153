
\chapter{Evolution equations of incompressible 2D fluids} 
%
%
\section{Classical non-rotating case} 
\label{sec_2Dcase}
%
The dimensional evolution equations of incompressible
homogeneous 2D-fluids on the plane (see e.g.\ \cite{batchelor1967}
or \cite{canutoetal1988}) in vorticity velocity form
\begin{equation} \label{eq_2Dvortvel}
  \zeta_{t} + \mathbf{u} \cdot \nabla \zeta = F + D,
\end{equation}
with $\mathbf{u} = \left( u,v \right)$ the vector of velocity fields 
in $x$ and $y$-direction, $F$ a forcing and $D$ a dissipation term. 
From homogeneity and incompressibility of the fluid we get
\begin{equation} \label{eq_conti}
  \mathbf{\nabla \cdot} \rho \mathbf{u} 
    = 
  \rho \mathbf{\nabla \cdot u} = 0,
\end{equation} 
with $\rho$ the fluid density. Using this property we can introduce 
a volume (mass) streamfunction measuring the volume (mass) flux  
across an arbitrary line from the point $(x_{0},y_{0})$ to a point $(x,y)$ 
via the path integral
\begin{equation} \label{eq_psizetadef}
  \psi(x,y) - \psi(x_{0},y_{0}) = - \int_{(x_{0},y_{0})}^{(x,y)}
  \left[
   \left(
    \begin{array}{c}
     u  
     \\ 
     v
    \end{array}
   \right)
    \mathbf{\cdot}
   \left(
    \begin{array}{c}
     -dy  
      \\ 
      dx 
    \end{array}
   \right)
  \right].
\end{equation}
The minus sign in front of the integral just changes the direction
of positive massflux across the line and is chosen to make 
the classical streamfunction of 2D fluids compatible to 
the streamfunction (see e.g.\ \cite{danilovandgurarie2000}) 
typically used for 2D rotating fluids in geosciences. 
In terms of the streamfunction $\psi$ the integrated mass flux 
$\mathcal{M}$ across a line joining the points $(x_{0},y_{0})$ and 
$(x,y)$ is given by
\begin{equation} \label{eq_calM}
  \mathcal{M}(x_{0},y_{0}|x,y)
   = - \rho H_{0} 
       \left[\psi(x,y) -  \psi(x_{0},y_{0}) \right],
\end{equation}
with $H_{0}$ the depth of the fluid. At the same time
the streamfunction $\psi$ is connected to the velocity 
field $(u,v)$ and the (relative) 
vorticity $\zeta = v_{x} - u_{y}$ via
\begin{equation} \label{eq_psiuv}
  (u,v) = (- \psi_{y},\psi_{x}) \ \mbox{and} \ \ \zeta = \Delta \psi.
\end{equation}
Using the streamfunction $\psi$, the evolution equation (\ref{eq_2Dvortvel})
can be written in the form
\begin{equation} \label{eq_2Dvortstream}
  \zeta_{t} + J(\psi,\zeta) = F + D,
\end{equation}
with $J(\psi,\zeta) = \psi_{x} \zeta_{y} - \psi_{y} \zeta_{x}$ the Jacobian.
Since the velocity field is divergence free (equation \ref{eq_conti})
different representations of the Jacobian $J$ can be found. Using 
appropriate superpositions of different representations one can derive 
discrete analogs of the Jacobian $J$ which keep the symmetry 
property $J(\psi,\zeta) = -J(\zeta,\psi)$ and conserve the integrated 
vorticity $\zeta$, kinetic energy $\nabla \psi \cdot \nabla \psi $  and
enstrophy $\zeta^{2}$. In \cite{arakawa1966} such a discrete analog is 
derived for a finite difference numerical model. Keeping the notation of 
\cite{arakawa1966} the Jacobian $J$ of equation (\ref{eq_2Dvortstream}) 
is denoted by $J_{1}$. Starting from $J_{1}$ we can derive a second form 
of the Jacobian 
\begin{equation} \label{eq_Jacobi2} 
  J_{2}(\psi,\zeta) 
   = 
  J_{1}(\psi,\zeta) + \psi \left( \zeta_{xy} - \zeta_{yx} \right)
   = 
  \left(\psi  \zeta_{y} \right)_{x}
   - 
  \left(\psi  \zeta_{x} \right)_{y},
\end{equation}
which gives the evolution equation
\begin{equation} \label{eq_2DJacobi2} 
  \zeta_{t} + \left(\psi  \zeta_{y} \right)_{x}  
            - \left(\psi  \zeta_{x} \right)_{y} = F + D.
\end{equation}
Further we can derive a third form of Jacobian $J_{3}$ as follows 
\begin{equation} \label{eq_Jacobi3}
  J_{3}(\psi,\zeta) 
   = 
  J_{1}(\psi,\zeta) + \zeta \left( \psi_{xy} - \psi_{yx} \right)
   = 
  \left(\zeta  \psi_{x} \right)_{y}
   - 
  \left(\zeta  \psi_{y} \right)_{x},
\end{equation}
which inserted in the evolution equation yields the 2D fluid equations
in flux form
\begin{equation} \label{eq_2Dflux}
  \zeta_{t} + 
  \left(\zeta  \psi_{x} \right)_{y}
   - 
  \left(\zeta  \psi_{y} \right)_{x},
   = 
  \zeta_{t} + \nabla \cdot \left(\mathbf{u} \zeta \right) 
   = F + D.
\end{equation}
It is possible to derive a fourth form of the Jacobian $J_{4}$
which is a funciton of the field $v^{2} - u^{2})$ and 
$uv$. We start again from the first Jacobian $J_{1}$
\begin{equation} \label{eq_Jacobi4}
  J_{4}(\psi,\zeta) 
   = 
  J_{1}(\psi,\zeta) + 
   \left(2 \zeta - \psi_{x} \right) \ \left( \psi_{xy} - \psi_{yx} \right)
   = 
  \left(v^{2} - u^{2} \right)_{xy}  
   + 
  \left(uv \right)_{xx-yy},
\end{equation}
which leads to the evolution equation
\begin{equation} \label{eq_2DJacobi4}
  \zeta_{t}+ \left( v^{2} - u^{2} \right)_{xy} + \left( uv \right)_{xx - yy} 
   = 
  F + D.
\end{equation}
All four forms given above are equivalent so that mathemtically one form is
sufficient to catch all properties of the equations. Of course also other
forms can be derived and it is also possible to superpose serveral forms.   
The different forms get important if one tries to find discrete 
representations of the Jacobian in the numerical scheme. Each form leads 
to a discrete representation with different conservation and symmetry 
properties. For more details see the chapters on the conservation 
properties (\ref{sec_conprops}) and on the pseudo-spectral method 
(\ref{sec_evolfourier}). 
%
\section{Quasi-two-dimensional rotating case} \label{sec_quasi2Dcase}
%
Starting from the shallow water equation on the $\beta$-plane one can
derive (see e.g. \cite{danilovandgurarie2000}) an equation
describing a rotating barotropic quasi-two-dimensional fluid which is a
generalization of the 2D-equation (\ref{eq_2Dvortvel}). The shallow
water potential vorticity is defined by 
\begin{equation} \label{eq_qshallow}
  q_{s} = \frac{\zeta + f}{H},
\end{equation}
where $H(x,y,t)$ is the fluid depth and $f$ the planetary vorticity. 
Decomposing the fluid depth into a mean depth $H_{0}$, a constant 
bottom topography B(x,y) and a time-dependent
depth deviation $h(x,y,t)$ we can write $H(x,y,t) = H_{0} + h(x,y,t) - B(x,y)$. 
Inserting the decomposition of the fluid depth into the equation 
(\ref{eq_qshallow}) we can expand the potential vorticity
\begin{equation} \label{eq_qshallowdecomp}
  q_{s} 
   =
  \frac{\zeta + f}{H_{0} + h - B}
   =
  \frac{1}{H_{0}} \ \frac{\zeta + f}{1 + \Delta H / H_{0}}
   =
  \frac{1}{H_{0}} \ 
  \left( \zeta + f \right) \left(1 - \frac{\Delta H}{H_{0}} + \dots \right),
\end{equation}
with $\Delta H = h - B$. Keeping only linear terms we finally get the barotopic
vorticity
\begin{equation} \label{eq_qbaro}
  q = H_{0} q_{s} = \zeta - \frac{f}{H_{0}} \left(h - B\right) + f.
\end{equation}
Here we assume that the depth deviations $\Delta H = h - B$ are much smaller
than the mean depth $H_{0}$. For small Rossby numbers $\mathrm{Ro} = U/Lf$, 
with $U$ a typical horizontal velocity scale, $L$ a typical horizontal 
length scale of the fluid motion and $f$ the local coriolis parameter 
we can (see again e.g.\ \cite{danilovandgurarie2000}) introduce 
the streamfunction 
\begin{equation} \label{eq_psibaro}
  \psi(x,y,t) = \frac{g}{f} \ h(x,y,t),    
\end{equation}
with $g$ the gravity. Using this streamfunction and expanding the 
coriolis parameter linarly we can write the barotropic 
quasi-geostrophic (QG) potential vorticity (PV) 
on the $\beta$-plane (\ref{eq_qbaro}) as 
\begin{equation} \label{eq_qdef}
  q = \left( \nabla^{2}- \alpha^{2} \right) \psi 
   + f_{0} + \beta y + \frac{f_{0}}{H_{0}} \ B.
\end{equation}
The linear expansion of the Coriolis parameter 
$f(\varphi) = 2 \Omega \sin \varphi$ which is defined on a sphere 
with radius $a$ and rotation rate $\Omega$ leads to
\begin{equation} \label{eq_fbeta}
  f(\varphi_{0} +  \Delta \varphi)
   =
  f(\varphi_{0}) + f^{\prime}(\varphi_{0}) \Delta \ \varphi   
   = 
  f_{0} + \beta  y,  
\end{equation}
with $\varphi_{0}$ the central latitude and the meridional 
coordinate $y = a \Delta \varphi$ which is the linearization of 
the projection $y = a \sin(\Delta \varphi)$. The $\beta$-parameter 
is defined by $\beta = 2 \Omega \cos(\varphi_{0})/a$. The modification 
parameter $\alpha = 1/L_{\mathrm{R}}$ is connected to the Rossby-Obukhov 
radius of deformation $L_{\mathrm{R}} = \sqrt{g H_{0}}/f$. 
As in the $2$-dimensional case the streamfunction $\psi$ 
(remind the different definition) is related to the velocity field 
$(u,v)$ again (see also equation \ref{eq_psiuv})  via
\begin{equation} \label{eq_upsibaro}
  u = -\psi_{y} \ \ \ \mbox{and} \ \ \ v = \psi_{x}.
\end{equation}
In an unforced and non-dissipative fluid the QG PV is materially conserved
\begin{equation} 
  q_{t} + J(\psi,q) = 0. 
\end{equation}  
Using the linear approximation of the coriolis parameter (\ref{eq_fbeta})
and introducing again forcing and dissipation we can write the evolution
equation for a barotropic fluid on the $\beta$-plane in the form
\begin{equation} \label{eq_quasi2Dbaro}
  q_{t} + J(\psi,q + \frac{f_{0}}{H_{0}} \ B) + \beta \psi_{x} = F + D,
\end{equation}
with the vorticity $q$ again given by
\begin{equation} \label{eq_vortquasi2Dbaro}
  q = \left(\nabla^2 -\alpha^{2} \right) \psi 
    = \zeta - \alpha^{2} \psi.
\end{equation}
Here we used the property of the Jacobian $J(f,f) = 0$ for all fields
$f(x,y)$ on the fluid domain.

Idealizing the bottom topography to a linear slope in y-direction 
$B(x,y) = B_{y} y$ the fluid experiences in addition to the ambient
planetary $\beta$-effect a so called topographic $\beta$-effect 
(see e.g.\ \cite{vanheist1994}) and the evolution equation simplifies to
\begin{equation} \label{eq_qbarotopobeta} 
  q_{t} + J(\psi,q) + \left(\beta + \frac{f_{0}}{H_{0}} B_{y} \right) \psi_{x} 
   = F + D.
\end{equation}
In the form (\ref{eq_quasi2Dbaro}) and (\ref{eq_vortquasi2Dbaro}) 
one can simulate incompressible 2D fluids and rotating barotropic 
quasi-2D fluids with the same set of equations using different parameters. 
In this more general frame the simplest case of a non-rotating
2D incompressible fluid is characterized by a vanishing 
ambient vorticity gradient, i.e.\ $\beta = 0$, and the limit of
an infinite Rossby radius $L_{\mathrm{R}} \longrightarrow \infty$
or a vanishing modification parameter $\alpha \longrightarrow 0$.
One has to keep in mind that the streamfunctions are different in 
the two cases (see e.g.\ \cite{johnstonandliu2004}) and that there 
are more subtle differences between 2D and QG Turbulence 
(see e.g.\ \cite{tungandorlando2003}). 
%
\section{Multi-layer quasi-geostrophic case} \label{sec_multilayerqg}
%
%
\section{The surface geostrophic case} \label{sec_sqg}
%
%
\section{Conservation properties} \label{sec_conprops}
%
The properties of the Jacobian and the conditions at the fluid boundaries
are at the base of the conservation properties of the fluid motions.
Given two functions $g(x,y)$ and $h(x,y)$ defined on the fluid domains
then the following integrals
\begin{equation} \label{eq_intjacobian01}
  \int_{x=0}^{X} \int_{x=0}^{Y} J(A,B) \ dxdy 
  , \
  \int_{x=0}^{X} \int_{x=0}^{Y} A J(A,B) \ dxdy 
   \ \ \mbox{and} \ \    
  \int_{x=0}^{X} \int_{x=0}^{Y} B J(A,B) \ dxdy 
\end{equation}
can be transformed through partial integration.



Above results can be generalized to more general fluid domains 
(see e.g.\ \cite{salmonandtalley1989}).


     


%
\section{Non-adiabatic terms}
%
\subsection{Laplacian based Viscosity and friction}

Internal viscosity and external friction of the fluid are described by 
the dissipation term $D$ on the left hand side of the equation 
(\ref{eq_quasi2Dbaro}). A classical way - the default reference in our
 model - to describe this term is to use a linear operator wich is a
superposition of powers of the Laplacian. 
The default dissipation in the fluid simulator is defined by
\begin{equation} \label{eq_Laplace_dissip}
  D \ q = - \left[\sigma \left(-1 \right)^{p_{\sigma}} \Delta^{p_{\sigma}}
                     +
                  \lambda \left(-1 \right)^{p_{\lambda}} \Delta^{p_{\lambda}}
            \right] q.
\end{equation}
Introducing dissipation time and length-scales we can write the 
dissipation operator also in the form 
\begin{equation} \label{eq_Laplace_dissip_02}
  D \ q = - \left[ 
              \frac{\left(-1 \right)^{p_{\sigma}}}{t_{\sigma}} 
              \left( 
               \frac{L_{\sigma}}{2 \pi}
              \right)^{2 p_{\sigma}}
              \Delta^{p_{\sigma}}
               +
              \frac{\left(-1 \right)^{p_{\lambda}}}{t_{\lambda}} 
              \left( 
                \frac{L_{\lambda}}{2 \pi}
              \right)^{2 p_{\lambda}}
              \Delta^{p_{\lambda}}
            \right] q.
\end{equation}
Here $L_{\sigma}$ and $L_{\lambda}$ are the small and 
large-scale cut-off length scales. The corresponding small and 
large-scale "damping" time scales are given by $t_{\sigma}$ 
and $t_{\lambda}$. The powers $p_{\sigma}$ of small-scale 
viscosity are in the range of $p_{\sigma} \in [1,2,3, \ \dots \ ]$ 
and the powers $p_{\lambda}$ of large-scale friction are in the 
range $p_{\lambda} \in [0,-1,-2,-3, \ \dots \ ]$. 
For $p_{\sigma} = 1$ and $p_{\lambda} = 0$ we speak of viscosity
and linear drag (friction), for $p_{\sigma} > 1$ and $p_{\lambda} < 0$
of hyperviscosity and hypofriction (see also \cite{danilovandgurarie2001}).
From equation (\ref{eq_Laplace_dissip}) it follows that the coefficients
$\sigma$ and $\lambda$ can be written by
\begin{equation} \label{eq_siglam}
  \sigma  = \left(\frac{L_{\sigma}}{2 \pi} \right)^{2 p_{\sigma}}
            \ \frac{1}{t_{\sigma}} 
          = \left(\frac{1}{r_{k,\sigma}}\right)^{2 p_{\sigma}}
            \ \frac{1}{t_{\sigma}} 
  \ \ \mbox{and} \ \
  \lambda = \left(\frac{L_{\lambda}}{2 \pi} \right)^{2 p_{\lambda}}
            \ \frac{1}{t_{\lambda}}
          = \left(\frac{1}{r_{k,\lambda}} \right)^{2 p_{\lambda}}
            \ \frac{1}{t_{\lambda}}.
\end{equation}
We have chosen the additional factor of $2 \pi$ since finally the
fluid domain is rescaled to multiples of $2 \pi$. Using this scaling
the coefficients $\sigma$ and $\lambda$ are characterized respectively 
by the damping the time scales $t_{\sigma}$ and $t_{\lambda}$ as well
as the cut-off wave number radius $r_{k,\sigma}$ and $r_{k,\lambda}$,
with $r_{k} = \sqrt{k^{2}_{x} + k^{2}_{y}}$.

The above dissipation operator belongs to a class of dissipation operators 
which are polynomials with positive and negative powers of the Laplacian
\begin{equation} \label{eq_Laplace_dissip_poly}
  D q = \sum_{n = 1}^{n_{max}} D_{n} q,
  \ \mbox{with} \ \
  D_{n} q 
   = 
 -\left[
   \sigma_{n-1} \left(-1 \right)^{n-1} \ \Delta^{n-1}
   \ +
   \lambda_{n} \left(-1 \right)^{-n} \ \Delta^{-n}
  \right] \ q.
\end{equation}
In Fourier space (see section \ref{sec_evolfourier}) the dissipation 
operators of this class reduce to the multiplication with polynomials 
in positive and negative powers of the wave numbers. More general 
dissipation operators can be constructed directly in Fourier space 
(see section \ref{sec_dissipation}).

\subsection{Forcing}
The forcing term $F$ in equations (\ref{eq_2Dvortvel}) and 
(\ref{eq_quasi2Dbaro}) describes forcings due to either 
external processes as a wind-stress or a moving plate or
non-resolved internal processes as, e.g.\ baroclinic instability 
or diabatic heating. Both types of forcings can be described 
in physical or spectral space. For a constant external forcing, 
e.g.\ a wind stress or drag of a moving plate $(\tau^{u},\tau^{v})$ 
given in $[N/m^{2}]$ and acting in $x$ and $y$ direction at the surface 
of the fluid the forcing term is given by 
\begin{equation} \label{eq_Fstressdrag}
 F(x,y) = \frac{1}{\rho H_{0}} \left( \tau_{x}^{v} - \tau_{y}^{u} \right),
\end{equation}
where in the rotating quasi-2D case the height deviations $h$ of the
fluid are neglected. The forcing can also be defined in spectral
space, see section \ref{sec_forcing} below.
%---
\section{Geometry and boundary conditions}
%
\subsection{Vertical boundary conditions}
%
The default lower boundary condition is a free-slip flat bottom.
Introducing a scale-independent damping corresponds to a bottom drag.
In the quasi-two-dimensional rotating case For fast rotating cases

By adding 

Further it is possible to introduce  
(Bretherton Haidvogel)

      At the bottom of the fluid one can introduce a 
%
\subsection{Horizontal boundary conditions}
%
{\bf Doubly periodic boundary conditions:} The evolution equations have to be completed by horizontal boundary 
conditions. The default geometry of the fluid domain is a square 
with an edge of length $L_{x} = L_{y} = L$ and the default boundary 
conditions are doubly periodic, i.e.\ $(f(x,y) = f(x+L,y+L)$ for all 
functions $f$ on the fluid domain. \\

\noindent
{\bf Channel boundary conditions:} In $x$-direction (zonal direction) 
we have periodic boudary conditions, i.e.\ $f(x,y) = f(x+L,y)$. 
In $y$-direction (meridional direction) at $y=0$ and $y=L$ we introduce 
walls with no-slip boundary conditions, i.e.\ the meridional velocity at 
the walls is zero $v(x,0) = v(x,L) =  0$. \\

\noindent
{\bf Box boundary condition:} In $x$-direction (zonal direction) 
we introduce walls with no-slip boundary conditions, i.e.\ the zonal 
velocity at the walls is zero $u(0,y) = u(L,y) = 0$. In $y$-direction 
(meridional direction) at $y=0$ and $y=L$ we introduce walls with 
no-slip boundary conditions, i.e.\ the meridional velocity at the walls 
is zero $v(x,0) = v(x,L) =  0$.
